\listfiles
\documentclass[final]{siamltex}
%\documentclass[draft]{siamltex}
\usepackage{showkeys}
\usepackage{amssymb,amsmath}

%%%%%%%%%%%%%%%%%%%%%%%%%%%%%%%%%%%%%%%%%%%%%%%%%%
% tikz
%%%%%%%%%%%%%%%%%%%%%%%%%%%%%%%%%%%%%%%%%%%%%%%%%%
\usepackage{
  tikz,%
  pgfplots, %
  xcolor, %
  mathdots, %
  %a4wide,
}
\pgfplotsset{compat=1.15}
\usetikzlibrary{calc}
\usetikzlibrary{arrows}
\usetikzlibrary{patterns}
\usetikzlibrary{decorations.pathmorphing}
\usepgfplotslibrary{statistics}
% \usetikzlibrary{fit}
% \usetikzlibrary{backgrounds}
% \usetikzlibrary{matrix}
% \usetikzlibrary{decorations.pathreplacing} %

\pgfplotscreateplotcyclelist{warmercolors}{
  {SPECblue!80!SPECorange},
  {SPECblue!70!SPECorange},
  {SPECblue!60!SPECorange},
  {SPECblue!50!SPECorange},
  {SPECblue!40!SPECorange},
  {SPECblue!30!SPECorange},
  {SPECblue!20!SPECorange},
  {SPECblue!10!SPECorange},
  {SPECorange},
  {SPECorange!90!SPECred},
  {SPECorange!80!SPECred},
  {SPECorange!70!SPECred},
  {SPECorange!60!SPECred},
  {SPECorange!50!SPECred},
  {SPECorange!40!SPECred},
  {SPECorange!30!SPECred},
  {SPECorange!20!SPECred},
  {SPECorange!10!SPECred},
  {SPECred},
  {SPECred!90!black},
  {SPECred!80!black},
  {SPECred!70!black},
  {SPECred!60!black},
  {SPECred!50!black},
  {SPECred!40!black},
  {SPECred!30!black},
  {SPECred!20!black},
  {SPECred!10!black},
  {black},
}

% some color definitions
\definecolor{SPECorange}{rgb}{1.0,.5625,0}
\definecolor{SPECblue}{rgb}{0,0,0.75}
\definecolor{SPECred}{rgb}{0.75,0,0}
\definecolor{SPECgreen}{rgb}{0,0.75,0}
\definecolor{SPECblack}{rgb}{0.75,0.75,0.75}
\definecolor{SPECCorange}{rgb}{1,0.9,0.6}
\definecolor{SPECCblue}{rgb}{0,0,0.25}
\definecolor{SPECCred}{rgb}{1.0,0,0}
\definecolor{SPECCgreen}{rgb}{0.25,0.75,0.25}

%\usepackage{arydshln}
\usepackage{cite} % Orders citations and puts 3-6, instead of 3,4,5,6
%\usepackage{rotfig_tikz} % rotfig with TikZ extension

% \usepackage{subfig}
% \usepackage{verbatim}
\usepackage{booktabs}
\usepackage{multirow}
\usepackage{listings}
%\usepackage{showkeys}
\usepackage[ruled, lined, boxed, algo2e]{algorithm2e}

%\theoremstyle{remark}
\newtheorem{Proposition}[theorem]{Proposition}
\newtheorem{remark}[theorem]{\it Remark}
\newtheorem{example}[theorem]{\it Example}


%%%%%%%%%%%%%%%%%%%%%%%%%%%%%%%%%%%%%%%%%%%%%%%%%%
\usepackage{ifpdf}
\ifpdf
\usepackage[pdftex,colorlinks=true,linkcolor=SPECred,filecolor=SPECgreen,citecolor=SPECblue,pdfpagemode=UseNone,hypertexnames=false]{hyperref}
\else
\usepackage[dvips,colorlinks=true,linkcolor=SPECred,filecolor=SPECgreen,citecolor=SPECblue,pdfpagemode=UseNone]{hyperref}
\fi
%%%%%%%%%%%%%%%%%%%%%%%%%%%%%%%%%%%%%%%%%%%%%%%%%%


%%%%%%%%%%%%%%%%%%%%%%%%%%%%%%%%%%%%%%%%%%%%%%%%%%
% define some commands
%%%%%%%%%%%%%%%%%%%%%%%%%%%%%%%%%%%%%%%%%%%%%%%%%%
%\newcommand{\absval}[1]{\mbox{$\mid \!#1\! \mid$}}
\providecommand{\abs}[1]{\left\lvert#1\right\rvert} %
\newcommand{\norm}[1]{\mbox{$\parallel\!#1\!\parallel$}}
\newcommand{\diagg}[1]{\mbox{diag$\left\{#1\right\}$}}
\newcommand{\ess}{\mbox{$\mathcal{S}$}}
%lfs\newcommand{\eee}{\mbox{$\mathcal{E}$}}
\newcommand{\kay}{\mbox{$\mathcal{K}$}}
\newcommand{\cue}{\mbox{$\mathcal{Q}$}}
\newcommand{\hv}[1]{\left[#1\right]}

%\renewcommand{\C}{\ensuremath{\mathbb{C}}}
\newcommand{\Cn}{\ensuremath{\mathbb{C}^{n}}}
\newcommand{\Cnn}{\ensuremath{\mathbb{C}^{n\times n}}}
\newcommand{\R}{\ensuremath{\mathbb{R}}}
\newcommand{\Rn}{\ensuremath{\mathbb{R}^{n}}}
\newcommand{\Rnn}{\ensuremath{\mathbb{R}^{n\times n}}}
\newcommand{\cO}{\ensuremath{\mathcal{O}}}
\newcommand{\cE}{\ensuremath{\mathcal{E}}}
\newcommand{\cH}{\ensuremath{\mathcal{H}}}
\newcommand{\cI}{\ensuremath{\mathcal{I}}}
\newcommand{\ctC}{\ensuremath{\mathcal{\widetilde{C}}}}
\newcommand{\tc}{\ensuremath{\widetilde{c}}}

\newcommand\oc{\ensuremath{\overline{c}}}
\newcommand\od{\ensuremath{\overline{d}}}
\newcommand\og{\ensuremath{\overline{g}}}
\newcommand\orr{\ensuremath{\overline{r}}}
\newcommand\os{\ensuremath{\overline{s}}}
\newcommand\ot{\ensuremath{\overline{t}}}
\newcommand\ox{\ensuremath{\overline{x}}}

\newcommand{\myspan}[1]{\ensuremath{\mathop{\mathrm{span}}\left\lbrace #1 \right\rbrace}}

\newcommand{\absval}[1]{\mbox{$\mid\!#1\!\mid$}}
\newcommand{\float}[1]{\mbox{fl$\left(#1\right)$}}
\newcommand{\spn}[1]{\mbox{{\rm span}$\left\{#1\right\}$}}
\newcommand{\cee}{\mbox{$\mathcal{C}$}}
\newcommand{\eee}{\mbox{$\mathcal{E}$}}
\newcommand{\aich}{\mbox{$\mathcal{H}$}}


%% bold math 
\newcommand{\bA}{\ensuremath{\mathbf{A}}}
\newcommand{\bB}{\ensuremath{\mathbf{B}}}
\newcommand{\bC}{\ensuremath{\mathbf{C}}}
\newcommand{\bD}{\ensuremath{\mathbf{D}}}
\newcommand{\bE}{\ensuremath{\mathbf{E}}}
\newcommand{\bF}{\ensuremath{\mathbf{F}}}
\newcommand{\bG}{\ensuremath{\mathbf{G}}}
\newcommand{\bH}{\ensuremath{\mathbf{H}}}
\newcommand{\bI}{\ensuremath{\mathbf{I}}}
\newcommand{\bJ}{\ensuremath{\mathbf{J}}}
\newcommand{\bK}{\ensuremath{\mathbf{K}}}
\newcommand{\bL}{\ensuremath{\mathbf{L}}}
\newcommand{\bM}{\ensuremath{\mathbf{M}}}
\newcommand{\bN}{\ensuremath{\mathbf{N}}}
\newcommand{\bO}{\ensuremath{\mathbf{O}}}
\newcommand{\bP}{\ensuremath{\mathbf{P}}}
\newcommand{\bQ}{\ensuremath{\mathbf{Q}}}
\newcommand{\bR}{\ensuremath{\mathbf{R}}}
\newcommand{\bS}{\ensuremath{\mathbf{S}}}
\newcommand{\bT}{\ensuremath{\mathbf{T}}}
\newcommand{\bU}{\ensuremath{\mathbf{U}}}
\newcommand{\bV}{\ensuremath{\mathbf{V}}}
\newcommand{\bW}{\ensuremath{\mathbf{W}}}
\newcommand{\bX}{\ensuremath{\mathbf{X}}}
\newcommand{\bY}{\ensuremath{\mathbf{Y}}}
\newcommand{\bZ}{\ensuremath{\mathbf{Z}}}

\newcommand{\ba}{\ensuremath{\mathbf{a}}}
\newcommand{\bb}{\ensuremath{\mathbf{b}}}
\newcommand{\bc}{\ensuremath{\mathbf{c}}}
\newcommand{\boldd}{\ensuremath{\mathbf{d}}}
\newcommand{\bolde}{\ensuremath{\mathbf{e}}}
\newcommand{\boldf}{\ensuremath{\mathbf{f}}}
\newcommand{\bg}{\ensuremath{\mathbf{g}}}
\newcommand{\bh}{\ensuremath{\mathbf{h}}}
\newcommand{\bi}{\ensuremath{\mathbf{i}}}
\newcommand{\bj}{\ensuremath{\mathbf{j}}}
\newcommand{\bk}{\ensuremath{\mathbf{k}}}
\newcommand{\bl}{\ensuremath{\mathbf{l}}}
\newcommand{\bm}{\ensuremath{\mathbf{m}}}
\newcommand{\bn}{\ensuremath{\mathbf{n}}}
\newcommand{\bo}{\ensuremath{\mathbf{o}}}
\newcommand{\bp}{\ensuremath{\mathbf{p}}}
\newcommand{\bq}{\ensuremath{\mathbf{q}}}
\newcommand{\br}{\ensuremath{\mathbf{r}}}
\newcommand{\bs}{\ensuremath{\mathbf{s}}}
\newcommand{\bt}{\ensuremath{\mathbf{t}}}
\newcommand{\bu}{\ensuremath{\mathbf{u}}}
\newcommand{\bv}{\ensuremath{\mathbf{v}}}
\newcommand{\bw}{\ensuremath{\mathbf{w}}}
\newcommand{\bx}{\ensuremath{\mathbf{x}}}
\newcommand{\by}{\ensuremath{\mathbf{y}}}
\newcommand{\bz}{\ensuremath{\mathbf{z}}}

\newcommand{\cN}{\ensuremath{\mathcal{N}}}

\newcommand{\ur}{\varepsilon_m} % UNIT ROUND-off

\newcommand{\lapack}{\textsf{LAPACK}}
\DeclareMathOperator{\expm}{\text{\rm expm}}

%\newcommand{\rr}[1]{\textbf{Writing Remark: #1}}
%\hyphenation{split-one}
\newcommand{\dmu}{\,\mathrm{d}\mu}

%%%%%%%%%%%%%%%%%%%%%%%%%%%%%%%%%%%%%%%%%%%%%%%%%%
% PGF Graphic Externalization
%%%%%%%%%%%%%%%%%%%%%%%%%%%%%%%%%%%%%%%%%%%%%%%%%%
\pgfrealjobname{plot}
\newif\ifexternalize
\externalizetrue
%\externalizefalse

%\newenvironment{tm}{\begin{quote}\small\sf{\color{SPECorange}$\star$~Thomas~}}{\end{quote}}


%%%%%%%%%%%%%%%%%%%%%%%%%%%%%%%%%%%%%%%%%%%%%%%%%%%%%%%%%%%%%%%%%%%%%%%%%%%%%%%%
% Start Document
% Frontpage
%%%%%%%%%%%%%%%%%%%%%%%%%%%%%%%%%%%%%%%%%%%%%%%%%%%%%%%%%%%%%%%%%%%%%%%%%%%%%%%%

\title{Solving the Parametric Eigenvalue Problem by Taylor Series and Chebyshev Expansion%
  \thanks{The research has been partially funded by the Deutsche
Forschungsgemeinschaft (DFG)---Project-ID 318763901---SFB1294.}
}

\author{Thomas Mach\footnotemark[3]\and Melina A.\ Freitag\footnotemark[3]}

\begin{document}
\maketitle

\renewcommand{\thefootnote}{\fnsymbol{footnote}}

\footnotetext[3]{%
  University of Potsdam, Institute of Mathematics, 
  Karl-Liebknecht-Str.\ 24--25, 14476 Potsdam, Germany;
  % Datenassimilation, Insitut f\"ur Mathematik, Univerist\"at Potsdam;\\
  \mbox{(\texttt{\{thomas.mach,melina.freitag\}@uni-potsdam.de})}.}%



\renewcommand{\thefootnote}{\arabic{footnote}}


\date{\today}
\maketitle

%%%%%%%%%%%%%%%%%%%%%%%%%%%%%%%%%%%%%%%%%%%%%%%%%%%%%%%%%%%%%%%%%%%%%%%%%%%%%%%%
% Abstract
%%%%%%%%%%%%%%%%%%%%%%%%%%%%%%%%%%%%%%%%%%%%%%%%%%%%%%%%%%%%%%%%%%%%%%%%%%%%%%%%
\begin{abstract}
  We discuss two approaches to solving the parametric (or stochastic) eigenvalue
  problem. One of them uses a Taylor expansion and the other a Chebyshev
  expansion. The parametric eigenvalue problem assumes that the matrix $A$
  depends on a parameter $\mu$, where $\mu$ might be a random
  variable. Consequently, the eigenvalues and eigenvectors are also functions of
  $\mu$. We compute a Taylor approximation of these functions about $\mu_{0}$ by
  iteratively computing the Taylor coefficients. The complexity of this approach
  is $O(n^{3})$ for all eigenpairs, if the derivatives of $A(\mu)$ at $\mu_{0}$
  are given.  The Chebyshev expansion works similarly. We first find an initial
  approximation iteratively which we then refine with Newton's method.  This
  second method is more expensive but provides a good approximation over the
  whole interval of the expansion instead around a single point.
  
  We present numerical experiments confirming the complexity and demonstrating
  that the approaches are capable of tracking eigenvalues at intersection
  points.  Further experiments shed light on the limitations of the Taylor
  expansion approach with respect to the distance from the expansion point
  $\mu_{0}$.

%% HTML/Unicode Version  
% We discuss two approaches to solving the parametric (or stochastic) eigenvalue
% problem. One of them uses a Taylor expansion and the other a Chebyshev
% expansion. The parametric eigenvalue problem assumes that the matrix A depends
% on a parameter μ, where μ might be a random variable. Consequently, the
% eigenvalues and eigenvectors are also functions of μ. We compute a Taylor
% approximation of these functions about μ₀ by iteratively computing the Taylor
% coefficients. The complexity of this approach is O(n³) for all eigenpairs, if
% the derivatives of A(μ) at μ₀ are given.  The Chebyshev expansion works
% similarly. We first find an initial approximation iteratively which we then
% refine with Newton's method.  This second method is more expensive but
% provides a good approximation over the whole interval of the expansion instead
% around a single point.
  
% We present numerical experiments confirming the complexity and demonstrating
% that the approaches are capable of tracking eigenvalues at intersection
% points.  Further experiments shed light on the limitations of the Taylor
% expansion approach with respect to the distance from the expansion point μ₀.
%% GAMM
% We discuss two approaches to solving the parametric (or stochastic) eigenvalue
% problem A(μ)λ(μ)=A(μ)v(μ). One of them uses a Taylor series expansion and the
% other a Chebyshev expansion. The parametric eigenvalue problem assumes that
% the matrix A depends on a parameter μ, where μ is real and might be a random
% variable. Consequently, the eigenvalues and eigenvectors are also functions of
% μ. Parametric eigenvalue problems occur in different applications, for
% instance when Hadamard matrix functions are used, for PDEs with unknown
% material constants, or in conjunction with matrix-value ODEs.

% We compute a Taylor approximation of these functions about μ₀ by iteratively
% computing the Taylor coefficients. The complexity of this approach is O(n³)
% for all eigenpairs, if the derivatives of A(μ) at μ₀ are given.  The Chebyshev
% expansion works similarly. We first find an initial approximation iteratively
% which we then refine with Newton's method.  This second method is more
% expensive but provides a good approximation over the whole interval of the
% expansion instead around a single point.
 
% We present numerical experiments confirming the complexity and demonstrating
% that the approaches are capable of tracking eigenvalues at intersection
% points.  Further experiments shed light on the limitations of the Taylor
% expansion approach with respect to the distance from the expansion point μ₀.
 \end{abstract}

%%%%%%%%%%%%%%%%%%%%%%%%%%%%%%%%%%%%%%%%%%%%%%%%%%%%%%%%%%%%%%%%%%%%%%%%%%%%%%%%
% Keywords
%%%%%%%%%%%%%%%%%%%%%%%%%%%%%%%%%%%%%%%%%%%%%%%%%%%%%%%%%%%%%%%%%%%%%%%%%%%%%%%%
\begin{keywords}
%\textit{incomplete list of keywords}, %
stochastic eigenvalue problem, %
parametric eigenvalue problem, %
Taylor expansion, %
Chebyshev expansion
\end{keywords}

%%%%%%%%%%%%%%%%%%%%%%%%%%%%%%%%%%%%%%%%%%%%%%%%%%%%%%%%%%%%%%%%%%%%%%%%%%%%%%%%
% AMS Subject Classification
%%%%%%%%%%%%%%%%%%%%%%%%%%%%%%%%%%%%%%%%%%%%%%%%%%%%%%%%%%%%%%%%%%%%%%%%%%%%%%%%
\begin{AMS}
  65F15% Eigenvalues, eigenvectors
  , 65H17% Eigenvalues, eigenvectors
  , 15A18% Eigenvalues, singular values, and eigenvectors
  % , 15A23% Factorization of matrices
  % , 15B10% Orthogonal matrices
  , 93B60% Systems and Control Theory -- Eigenvalue problems
\end{AMS}

%%%%%%%%%%%%%%%%%%%%%%%%%%%%%%%%%%%%%%%%%%%%%%%%%%%%%%%%%%%%%%%%%%%%%%%%%%%%%%%%
% Headings
%%%%%%%%%%%%%%%%%%%%%%%%%%%%%%%%%%%%%%%%%%%%%%%%%%%%%%%%%%%%%%%%%%%%%%%%%%%%%%%%
\pagestyle{myheadings} %
\thispagestyle{plain} %
\markboth{T.\ Mach and M. A. Freitag
}%
{Solving Parametric Eigenvalue Problems}

\section{Introduction}
\label{sec:introduction}



\input{exp_pap_fig52_1000_26_0.200000_1_20}

%exp_pap_fig52_8_26_0.200000_1.tex

\end{document}

%% Thomas' emacs local variable set up

%%% Local Variables: 
%%% mode: LaTeX
%%% TeX-PDF-mode:t
%%% TeX-engine: luatex
%%% auto-fill-function:nil
%%% mode:auto-fill
%%% flyspell-mode:nil
%%% eval: (flyspell-mode)
%%% ispell-local-dictionary: "american"
%%% End: 









